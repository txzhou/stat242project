%% LyX 2.1.3 created this file.  For more info, see http://www.lyx.org/.
%% Do not edit unless you really know what you are doing.
\documentclass[english]{article}
\usepackage[T1]{fontenc}
\usepackage[latin9]{inputenc}
\usepackage{geometry}
\geometry{verbose,tmargin=1in,bmargin=1in,lmargin=1in,rmargin=1in}
\setlength{\parskip}{\medskipamount}
\setlength{\parindent}{0pt}
\usepackage{setspace}
\doublespacing
\usepackage{babel}
\begin{document}

\title{Project Proposal}


\author{Jacob Humber, Michael Levy and Tianxia Zhou}

\maketitle

\section*{Overview}

We propose to use Shiny and USGS's dataRetrieval to create a webpage
where users can interactively visualize water data such as streamflow
rates, reservoir volume, and groundwater levels. 


\section*{Details}

Shiny is an R package to create interactive plots. Shiny applications
consist of a user-interface script for the front-end and a server
script for the back-end. We envision a front-end where users can select
data by place (e.g. stream, groundwater basin), time, and plot type
and dimensions. We need to explore the details of the data before
determining precisely what plotting options will be available, but
we envision some mapping options as well as time series and comparative
plots. Specifically, we plan to construct visualizations which depict
the current status the drought within California.  The server side
of the Shiny app will leverage USGS's dataRetrieval tool to query
data at the time of the user's request. This will allow us to make
USGS's extensive database available to users with minimal storage
requirements on our end. 

The aforementioned R package dataRetrieval contains numerous functions
which query hydrologic data directly from the USGS's National Water
Information System (NWIS). The NWIS contains a myriad of data for
both surface water, groundwater as well as water use. Additionally,
dataRetrieval provides access to water quality data from the Water
Quality Portal. 
\end{document}
