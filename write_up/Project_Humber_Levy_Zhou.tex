%% LyX 2.1.3 created this file.  For more info, see http://www.lyx.org/.
%% Do not edit unless you really know what you are doing.
\documentclass[english]{article}
\usepackage[latin9]{inputenc}
\usepackage{geometry}
\geometry{verbose,tmargin=1in,bmargin=1in,lmargin=1in,rmargin=1in}
\setlength{\parskip}{\medskipamount}
\setlength{\parindent}{0pt}
\usepackage{setspace}
\doublespacing
\usepackage{babel}
\begin{document}

\title{Project: Visualization of California Water}

\maketitle
\begin{center}
Git Repository: https://z109620@bitbucket.org/z109620/project.git
\par\end{center}

\noindent \begin{center}
Visualization: 
\par\end{center}

\noindent \begin{center}
Jacob Humber, Michael Levy and Tianxia Zhou
\par\end{center}


\section{Introduction}

California has been experiencing a drought since the onset of 2012.
Over this time LIST BAD SHIT THAT HAPPENED. In an effort to convey
the magnitude of this persistent drought the United States Geological
Survey (USGS) generated an interesting visualization\footnote{The USGS visualization: http://cida.usgs.gov/ca\_drought/}.
While this visualization helps to illuminate the magnitude of the
California drought, it focuses almost exclusively on surface water
reservoir levels. Surface water reservoir levels only tell part of
the story of the California drought. Clearly, an thorough understanding
of California water consumption, stream flows, the depth of ground
water wells are all important. The current project, therefore, will
generate visualizations\footnote{Our visualization: INSERT THE ADDRESS TO OUR VISUALIZATION}
which will depict each of these aforementioned water metrics and is
therefore meant to compliment the existing USGS visualization. 

All visualizations within the project are generated with the R package
shiny. For a proficient R user utilizing the shiny package is relatively
straightforward. At a minimum a shiny application consists of a user-interface
(ui.R) script for the front-end and a server script (server.R) for
the back-end. Each of these scripts are written in R. Consequently,
the shiny packages provides some of the elegance of JavaScript without
needing to know HTML, CSS or even JavaScript itself! 

Three distinct visualization are generated within the project. The
first depicts California water consumption by county and sector. The
next depicts discharge rates of streams and rivers. Finally, the last
visualization depicts the ground water depth of wells. In what follows,
we will discuss each of these visualization separately within sections
2, 3 and 4. 


\section{Water Consumption}

As water becomes scare, it is essential to understand both the industries
as well as the areas within California that consume the most water.
These sectors and areas are those most likely to bare the brunt of
the hardship that is the California drought.

The USGS visualization does provides a pie chart of water consumption
by industry. However, in the current project we expand this by depicting
the water consumption by county by year for 2000, 2005 and 2010\footnote{Data source: http://water.usgs.gov/watuse/data/}.
Unfortunately, the 2015 data is not currently available. Additionally
upon a click of a county, a graph appears which displays the water
consumption by sector for that county. 

Since the data set for water consumption is small, we store all the
data remotely on our Git Hub page. Intentionally we wanted to host
our shiny application on shinyapps.io, however, the only way to host
data on this website is to pay \$30 a month, which is a lot of money
for a group of graduate students.

The code to generate this visualization is contained the in R scripts
plot.R and functions.R. The color.map function within functions.R
generates the plot of California. This is essentially a wrapper of
the map function from R's maps package. Upon clicking on a county,
a graph of the consumption by sector for that county is generated
with the ggplot wrapper gg.wrapper. Notice that when a users clicks
the map, longitude and latitude coordinates are passed to server.R,
however we don't want to pass longitude and latitude coordinates to
the gg.wrapper, we want to pass a county name. Consequently, we must
convert these longitude and latitude coordinates to a county. This
is done with the latlong2county function within functions.R\footnote{The latlong2county function draws heavily from the following StackOverflow
thread: http://stackoverflow.com/questions/13316185 }.


\section{Stream flow}

In the current context stream flow is measured by the discharge rate.
The discharge rate is the volume of water moving down a stream or
river per unit of time, in our context the discharge rate is measured
in cubic feet per second.

The USGS visualization implicitly depicts stream flows insofar as
the different drought categories within their visualization - No Drought,
Abnormally Dri, Moderate Drought, Severe Drough, Extreme Drought and
Exceptional Drought - are defined by varying stream flows. However,
it is not possible to obtain, from their visualization, the actual
stream flow from participial streams. Consequently, our visualization
depicts the discharge rate read from each surface gauge monitored
by the USGS

In this context we have utilized the leafet function in the R package
of the same name to draw our map of California, see the R script server.R.
Like D3, leafet is a JavaScript library. The leafet package, which
is similar to shiny in this regard, introduces some of the functionally
of JavaScript without the need to know how to write JavaScript code.
The main reason for invoking leafet here is that this function makes
it straightforward to create a map which the user can zoom in and
out on. Given the numerous observation, to insure clarity, it was
necessary that user have the ability to zoom. Another interesting
feature of this visualization is that unlike the visualization of
water consumption, the data on discharge rates are not stored remotely
on Git Hub. This data is expansive and consequently we opted to load
data upon a click. This is made possible by the R package dataRetrieval.
This packages allows users to query the expansive hydrological data
provided by USGS. Consequently, when a user clicks a site, the corresponding
site number is passed to dataRetrieval. The data is in turn passed
to plot wrapper functions in USGSplot.R.


\section{Ground Water}

Ground water wells are not currently depicted within the USGS visualization.
Despite this fact, ground water is essential to California's water
supply as it comprises 30\% of the California water supply. Consequently,
we provided a visualization of water well depth for all wells currently
monitored by the USGS.

The approach used to generate the ground water visualization is very
similar to the stream flow visualization. As above both, the leafet
and dataRetrieval packages are utilized. However, one additional feature
of this visualization is that for each successive on a well sites
the well depth accumulate on the plot. This is helpful for comparisons
across wells. The action bottom below the plot, clears all but the
last plot. 


\section{Conclusion}

This project compliments the USGS visualization. It does so by visualizing
water consumption, stream flows and ground water elevation. In this
project we opt to use the R shiny package. Clearly, this assignment
could have been done using JavaScript, however, shiny is capable to
generate some very nice looking graphics and only requires a knowledge
of R. That withstanding, a fun summer project would be do redo some
of this project in JavaScript to understand the differences between
these two approaches. 

\newpage


\section*{\noindent Appendix}

ui.R

<<>>=
@

server.R

<<>>=
@

functions.R

<<>>=
@

plot.R

<<>>=
@

USGSplot.R

<<>>=
@

readUSGSData.R

<<>>=
@
\end{document}
