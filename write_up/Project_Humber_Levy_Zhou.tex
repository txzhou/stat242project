\batchmode
\makeatletter
\def\input@path{{E:/Desktop/project/write_up//}}
\makeatother
\documentclass[english]{article}\usepackage[]{graphicx}\usepackage[]{color}
%% maxwidth is the original width if it is less than linewidth
%% otherwise use linewidth (to make sure the graphics do not exceed the margin)
\makeatletter
\def\maxwidth{ %
  \ifdim\Gin@nat@width>\linewidth
    \linewidth
  \else
    \Gin@nat@width
  \fi
}
\makeatother

\definecolor{fgcolor}{rgb}{0.345, 0.345, 0.345}
\newcommand{\hlnum}[1]{\textcolor[rgb]{0.686,0.059,0.569}{#1}}%
\newcommand{\hlstr}[1]{\textcolor[rgb]{0.192,0.494,0.8}{#1}}%
\newcommand{\hlcom}[1]{\textcolor[rgb]{0.678,0.584,0.686}{\textit{#1}}}%
\newcommand{\hlopt}[1]{\textcolor[rgb]{0,0,0}{#1}}%
\newcommand{\hlstd}[1]{\textcolor[rgb]{0.345,0.345,0.345}{#1}}%
\newcommand{\hlkwa}[1]{\textcolor[rgb]{0.161,0.373,0.58}{\textbf{#1}}}%
\newcommand{\hlkwb}[1]{\textcolor[rgb]{0.69,0.353,0.396}{#1}}%
\newcommand{\hlkwc}[1]{\textcolor[rgb]{0.333,0.667,0.333}{#1}}%
\newcommand{\hlkwd}[1]{\textcolor[rgb]{0.737,0.353,0.396}{\textbf{#1}}}%

\usepackage{framed}
\makeatletter
\newenvironment{kframe}{%
 \def\at@end@of@kframe{}%
 \ifinner\ifhmode%
  \def\at@end@of@kframe{\end{minipage}}%
  \begin{minipage}{\columnwidth}%
 \fi\fi%
 \def\FrameCommand##1{\hskip\@totalleftmargin \hskip-\fboxsep
 \colorbox{shadecolor}{##1}\hskip-\fboxsep
     % There is no \\@totalrightmargin, so:
     \hskip-\linewidth \hskip-\@totalleftmargin \hskip\columnwidth}%
 \MakeFramed {\advance\hsize-\width
   \@totalleftmargin\z@ \linewidth\hsize
   \@setminipage}}%
 {\par\unskip\endMakeFramed%
 \at@end@of@kframe}
\makeatother

\definecolor{shadecolor}{rgb}{.97, .97, .97}
\definecolor{messagecolor}{rgb}{0, 0, 0}
\definecolor{warningcolor}{rgb}{1, 0, 1}
\definecolor{errorcolor}{rgb}{1, 0, 0}
\newenvironment{knitrout}{}{} % an empty environment to be redefined in TeX

\usepackage{alltt}
\usepackage[T1]{fontenc}
\usepackage[latin9]{inputenc}
\usepackage{geometry}
\geometry{verbose,tmargin=1in,bmargin=1in,lmargin=1in,rmargin=1in}
\setlength{\parskip}{\medskipamount}
\setlength{\parindent}{0pt}
\usepackage{setspace}
\doublespacing
\usepackage{babel}
\IfFileExists{upquote.sty}{\usepackage{upquote}}{}
\begin{document}

\title{Project: Visualization of California Water}

\maketitle
\begin{center}
Bit Bucket Repository: https://z109620@bitbucket.org/z109620/project.git
\par\end{center}

\noindent \begin{center}
R code to access our Visualization: require(shiny); runGitHub( repo
= \textquotedbl{}stat242project\textquotedbl{}, username = \textquotedbl{}txzhou\textquotedbl{},
subdir = \textquotedbl{}shinyApp\textquotedbl{})
\par\end{center}

\noindent \begin{center}
Jacob Humber, Michael Levy and Tianxia Zhou
\par\end{center}


\section{Introduction}

Since the onset of 2012, California has been experiencing one of its
worst droughts in over a thousand years\footnote{http://www.nytimes.com/2015/05/03/opinion/sunday/the-end-of-california.html?\_r=0}.
For the year of 2014 alone the agricultural losses as a results of
the drought totaled 2.2 billion dollars as well as a loss of 17,000
agricultural jobs \footnote{http://www.wsj.com/articles/drought-will-cost-california-2-2-billion-in-losses-costs-this-year-1405452120}.
In an effort to convey the magnitude of this persistent drought the
United States Geological Survey (USGS) generated an interesting visualization\footnote{The USGS visualization: http://cida.usgs.gov/ca\_drought/}.
While this visualization helps to illuminate the magnitude of the
California drought, it focuses almost exclusively on surface water
reservoir levels. Surface water reservoir levels only tell part of
the story of the California drought. Clearly, an thorough understanding
of California water consumption, stream flows, the depth of ground
water wells are all important. The current project, therefore, will
generate visualizations\footnote{Our visualization: To access the visualization the following lines
of code must be executed: require(shiny); runGitHub( repo = \textquotedbl{}stat242project\textquotedbl{},
username = \textquotedbl{}txzhou\textquotedbl{}, subdir = \textquotedbl{}shinyApp\textquotedbl{}).
Alternatively, the shiny app can be run by cloning the repository
locally on your own machine.} which will depict each of these aforementioned water metrics and
is therefore meant to compliment the existing USGS visualization. 

All visualizations within the project are generated with the R package
shiny. For a proficient R user utilizing the shiny package is relatively
straightforward. At a minimum a shiny application consists of a user-interface
(ui.R) script for the front-end and a server script (server.R) for
the back-end. Each of these scripts are written in R. Consequently,
the shiny packages provides some of the elegance of JavaScript without
needing to know HTML, CSS or even JavaScript itself! 

Three distinct visualization are generated within the project. The
first depicts California water consumption by county and sector. The
next depicts discharge rates of streams and rivers. Finally, the last
visualization depicts the ground water depth of wells. In what follows,
we will discuss each of these visualization separately within sections
2, 3 and 4. All R scripts are contained in the Appendix below.


\section{Water Consumption}

As water becomes scare, it is essential to understand both the industries
as well as the areas within California that consume the most water.
These sectors and areas are those most likely to bare the brunt of
the hardship that is the California drought.

The USGS visualization does provides a pie chart of water consumption
by industry. However, in the current project we expand this by depicting
the water consumption by county by year for 2000, 2005 and 2010\footnote{Data source: http://water.usgs.gov/watuse/data/}.
Unfortunately, the 2015 data is not currently available. Additionally
upon a click of a county, a graph appears which displays the water
consumption by sector for that county. 

Since the data set for water consumption is small, we store all the
data remotely on our Git Hub page. Intentionally we wanted to host
our shiny application on shinyapps.io, however, the only way to host
data on this website is to pay \$30 a month, which is a lot of money
for a group of graduate students.

The code to generate this visualization is contained the in R scripts
plot.R and functions.R in the Appendix. The color.map function within
functions.R generates the plot of California. This is essentially
a wrapper of the map function from R's maps package. Upon clicking
on a county, a graph of the consumption by sector for that county
is generated with the ggplot wrapper gg.wrapper. Notice that when
a users clicks the map, longitude and latitude coordinates are passed
to server.R, however we don't want to pass longitude and latitude
coordinates to the gg.wrapper, we want to pass a county name. Consequently,
we must convert these longitude and latitude coordinates to a county.
This is done with the latlong2county function within functions.R\footnote{The latlong2county function draws heavily from the following StackOverflow
thread: http://stackoverflow.com/questions/13316185 }.


\section{Stream flow}

In the current context stream flow is measured by the discharge rate.
The discharge rate is the volume of water moving down a stream or
river per unit of time, in our context the discharge rate is measured
in cubic feet per second.

The USGS visualization implicitly depicts stream flows insofar as
the different drought categories within their visualization - No Drought,
Abnormally Dri, Moderate Drought, Severe Drough, Extreme Drought and
Exceptional Drought - are defined by varying stream flows. However,
it is not possible to obtain, from their visualization, the actual
stream flow from participial streams. Consequently, our visualization
depicts the discharge rate read from each surface gauge monitored
by the USGS

In this context we have utilized the leafet function in the R package
of the same name to draw our map of California, see the R script server.R.
Like D3, leafet is a JavaScript library. The leafet package, which
is similar to shiny in this regard, introduces some of the functionally
of JavaScript without the need to know how to write JavaScript code.
The main reason for invoking leafet here is that this function makes
it straightforward to create a map which the user can zoom in and
out on. Given the numerous observation, to insure clarity, it was
necessary that user have the ability to zoom. Another interesting
feature of this visualization is that unlike the visualization of
water consumption, the data on discharge rates are not stored remotely
on Git Hub. This data is expansive and consequently we opted to load
data upon a click. This is made possible by the R package dataRetrieval.
This packages allows users to query the expansive hydrological data
provided by USGS. Consequently, when a user clicks a site, the corresponding
site number is passed to dataRetrieval. The data is in turn passed
to plot wrapper functions in USGSplot.R.


\section{Ground Water}

Ground water wells are not currently depicted within the USGS visualization.
Despite this fact, ground water is essential to California's water
supply as it comprises 30\% of the California water supply. Consequently,
we provided a visualization of water well depth for all wells currently
monitored by the USGS.

The approach used to generate the ground water visualization is very
similar to the stream flow visualization. As above both, the leafet
and dataRetrieval packages are utilized. 


\section{Conclusion}

This project compliments the USGS visualization. It does so by visualizing
water consumption, stream flows and ground water elevation. In this
project we opt to use the R shiny package. Clearly, this assignment
could have been done using JavaScript, however, shiny is capable to
generate some very nice looking graphics and only requires a knowledge
of R. That withstanding, a fun summer project would be do redo some
of this project in JavaScript to understand the differences between
these two approaches. 

\newpage


\section*{\noindent Appendix}

ui.R

\begin{knitrout}
\definecolor{shadecolor}{rgb}{0.969, 0.969, 0.969}\color{fgcolor}\begin{kframe}
\begin{alltt}
\hlkwd{library}\hlstd{(shiny)}
\hlkwd{library}\hlstd{(maps)}
\hlkwd{library}\hlstd{(leaflet)}
\hlkwd{library}\hlstd{(mapdata)}
\hlkwd{library}\hlstd{(maptools)}
\hlkwd{library}\hlstd{(Hmisc)}
\hlkwd{library}\hlstd{(ggplot2)}
\hlkwd{library}\hlstd{(reshape2)}
\hlkwd{library}\hlstd{(dataRetrieval)}
\hlkwd{library}\hlstd{(data.table)}
\hlkwd{library}\hlstd{(RColorBrewer)}

\hlcom{# Define UI for application}
\hlkwd{shinyUI}\hlstd{(}
  \hlkwd{navbarPage}\hlstd{(}
    \hlstr{"California Water Use"}\hlstd{,}

    \hlkwd{tabPanel}\hlstd{(}
      \hlstr{"Use by county"}\hlstd{,}
      \hlkwd{h1}\hlstd{(}\hlstr{"Water use"}\hlstd{),}

      \hlkwd{fluidRow}\hlstd{(}
        \hlkwd{column}\hlstd{(}\hlnum{5}\hlstd{,}
               \hlcom{#Selection box}
               \hlkwd{selectInput}\hlstd{(}\hlstr{"metric"}\hlstd{,}
                           \hlkwc{label} \hlstd{=} \hlstr{"Choose a Water Consumption Metric"}\hlstd{,}
                           \hlkwc{choices} \hlstd{=} \hlkwd{list}\hlstd{(}\hlstr{"Percent of California Consumption"}\hlstd{,}
                                          \hlstr{"Per Capita Consumption"}\hlstd{)),}

               \hlkwd{selectInput}\hlstd{(}\hlstr{"year"}\hlstd{,}
                           \hlkwc{label} \hlstd{=} \hlstr{"Choose a Year"}\hlstd{,}
                           \hlkwc{choices} \hlstd{=} \hlkwd{list}\hlstd{(}\hlnum{2000}\hlstd{,}\hlnum{2005}\hlstd{,}\hlnum{2010}\hlstd{)),}

               \hlkwd{helpText}\hlstd{(}\hlstr{"Click on a county for consumption by 
					sector for that county in the selected year."}\hlstd{),}

               \hlcom{# Print the clickable map}
               \hlkwd{plotOutput}\hlstd{(}\hlstr{"theMap"}\hlstd{,} \hlkwc{height} \hlstd{=} \hlstr{"400px"}\hlstd{,} \hlkwc{click} \hlstd{=} \hlstr{"plotclick"}\hlstd{),}

               \hlkwd{textOutput}\hlstd{(}\hlstr{"badCounty"}\hlstd{)}  \hlcom{# This line has to be present for the}
               \hlcom{# conditionalPanel()s to work. I have no idea why. Maybe badCounty has}
               \hlcom{# to be evaluated before the JS test is called???}


        \hlstd{),}

        \hlkwd{column}\hlstd{(}\hlnum{6}\hlstd{,}
               \hlkwd{helpText}\hlstd{(}\hlstr{"California is a geographically diverse state with widely varying 
                economies and population densities across its counties. Here you 
                can explore how much water each county uses and what they use it for."}\hlstd{),}

               \hlcom{# offset = 1,}
               \hlcom{# Show a plot of the generated distribution}
               \hlkwd{conditionalPanel}\hlstd{(}\hlstr{"output.badCounty == 0"}\hlstd{,}
                                \hlkwd{plotOutput}\hlstd{(}\hlstr{"useagePlot"}\hlstd{))}
        \hlstd{)}
      \hlstd{)}
    \hlstd{),}

    \hlkwd{tabPanel}\hlstd{(}
      \hlstr{"Surface water flow"}\hlstd{,}
      \hlkwd{h1}\hlstd{(}\hlstr{"Surface water"}\hlstd{),}

      \hlkwd{fluidRow}\hlstd{(}
        \hlkwd{column}\hlstd{(}\hlnum{5}\hlstd{,} \hlkwd{leafletOutput}\hlstd{(}\hlstr{"siteMap"}\hlstd{)),}
        \hlkwd{column}\hlstd{(}\hlnum{6}\hlstd{,}
               \hlkwd{helpText}\hlstd{(}\hlstr{"Surface water provides some of California's water supply. It is 
               stored in massive manmade reservoirs on almost all the major rivers and
               in the natural reservoir of high alpine snowpack in the Sierra
               Nevada mountains. In the current drought, snowpack has been far 
               less than normal, which means less runoff to fill the reservoirs, 
               the levels of which have dropped dramatically."}\hlstd{),}
               \hlkwd{br}\hlstd{(),}
               \hlkwd{helpText}\hlstd{(}\hlstr{"The map to left shows US Geological Survey surface water monitoring stations.
               Click on one to see how runoff changes seasonally and annually over
               the course of the drought. Try zooming in and looking at differences
               above and below major dams."}\hlstd{)}
        \hlstd{)}
      \hlstd{),}

      \hlcom{#        column(6, }
      \hlcom{#               offset = 1,}
      \hlkwd{conditionalPanel}\hlstd{(}\hlstr{"!is.na(output.mapClick)"}\hlstd{,}
                       \hlkwd{plotOutput}\hlstd{(}\hlstr{"sitePlot"}\hlstd{))}
      \hlcom{#        )}
      \hlcom{#      )}
    \hlstd{),}

    \hlkwd{tabPanel}\hlstd{(}
      \hlstr{"Groundwater levels"}\hlstd{,}
      \hlkwd{h1}\hlstd{(}\hlstr{"Groundwater"}\hlstd{),}

      \hlkwd{fluidRow}\hlstd{(}
        \hlkwd{column}\hlstd{(}\hlnum{5}\hlstd{,} \hlkwd{leafletOutput}\hlstd{(}\hlstr{"gwMap"}\hlstd{)),}
        \hlkwd{column}\hlstd{(}\hlnum{6}\hlstd{,}
               \hlkwd{helpText}\hlstd{(}\hlstr{"If surface water is California's first stop for water needs, 
               groundwater is its backstop. Only a few municipalities still
               rely on groundwater (Davis being one of them); however, it is of critical 
               importance for irrigation. When surface supplies are down, 
               reliance on groundwater increases, and unlike surface water and 
               unlike every other state in the US, groundwater use is unregulated
               in California. This has led to a furious groundwater grab during
               the current drought."}\hlstd{),}
               \hlkwd{br}\hlstd{(),}
               \hlkwd{helpText}\hlstd{(}\hlstr{"The map to the left shows the groundwater wells that the the US 
               Geological Survey uses to monitor groundwater elevation. Unfortunately,
               even public monitoring of groundwater levels is legally challenging
               in California, so data are sparser than hydrologists and water
               managers would like. However, here you can click on a series of wells 
               to see how groundwater levels have changed over the course of the 
               current drought."}\hlstd{)}
        \hlstd{)}
      \hlstd{),}

      \hlcom{#        column(6, }
      \hlcom{#               offset = 1,}
      \hlkwd{plotOutput}\hlstd{(}\hlstr{"GWPlot"}\hlstd{)}
      \hlcom{#        )}

      \hlcom{#      textOutput("wellsInfo"),}

      \hlcom{#      )}
    \hlstd{)}
  \hlstd{)}
\hlstd{)}
\end{alltt}
\end{kframe}
\end{knitrout}

server.R

\begin{knitrout}
\definecolor{shadecolor}{rgb}{0.969, 0.969, 0.969}\color{fgcolor}\begin{kframe}
\begin{alltt}
\hlcom{#path.app <- "C:/Users/Athena/Desktop/project/shinyApp/"}
\hlcom{#path.toapp <- "C:/Users/Athena/Desktop/project"}

\hlcom{#setwd(path.app)}

\hlcom{# source files ####}
\hlkwd{source}\hlstd{(}\hlkwc{file} \hlstd{=} \hlstr{"functions.R"}\hlstd{)}
\hlkwd{source}\hlstd{(}\hlkwc{file} \hlstd{=} \hlstr{"plot.R"}\hlstd{)}  \hlcom{# Modified this so it brings df.long into the workspace}
\hlkwd{source}\hlstd{(}\hlkwc{file} \hlstd{=} \hlstr{"readUSGSData.R"}\hlstd{)}
\hlkwd{source}\hlstd{(}\hlkwc{file} \hlstd{=} \hlstr{"USGSplot.R"}\hlstd{)}

\hlkwd{shinyServer}\hlstd{(}\hlkwa{function}\hlstd{(}\hlkwc{input}\hlstd{,} \hlkwc{output}\hlstd{) \{}

  \hlstd{theCounty} \hlkwb{=} \hlkwd{reactive}\hlstd{(\{}
    \hlkwa{if}\hlstd{(}\hlkwd{is.null}\hlstd{(input}\hlopt{$}\hlstd{plotclick))}
      \hlkwd{return}\hlstd{(}\hlstr{"none"}\hlstd{)}
    \hlkwd{latlong2county}\hlstd{(}
      \hlkwd{data.frame}\hlstd{(}\hlkwc{x} \hlstd{= input}\hlopt{$}\hlstd{plotclick}\hlopt{$}\hlstd{x,} \hlkwc{y} \hlstd{= input}\hlopt{$}\hlstd{plotclick}\hlopt{$}\hlstd{y))}
  \hlstd{\})}


  \hlcom{# Test whether theCounty() is valid, for startup and bad clicks, to not}
  \hlcom{# display charts on ui side. Idea from https://gist.github.com/ptoche/8312791}
  \hlstd{output}\hlopt{$}\hlstd{badCounty} \hlkwb{<-} \hlkwd{renderText}\hlstd{(\{}
    \hlkwa{if}\hlstd{(}
      \hlkwd{theCounty}\hlstd{()} \hlopt{==} \hlstr{"none"} \hlopt{|}
        \hlkwd{is.na}\hlstd{(}\hlkwd{theCounty}\hlstd{())}   \hlopt{|}
        \hlopt{!}\hlkwd{any}\hlstd{(} \hlkwd{grepl}\hlstd{(}\hlkwd{theCounty}\hlstd{(), counties) )}
    \hlstd{) \{}
      \hlkwd{return}\hlstd{(}\hlnum{1}\hlstd{)}
    \hlstd{\}} \hlkwa{else} \hlstd{\{}
      \hlkwd{return}\hlstd{(}\hlnum{0}\hlstd{)}
    \hlstd{\}}
  \hlstd{\})}

  \hlstd{output}\hlopt{$}\hlstd{theMap} \hlkwb{<-} \hlkwd{renderPlot}\hlstd{(\{}
    \hlcom{# color.variable == 1L or 2L (can include more variables)}
    \hlcom{# see functions.R for detail.}
    \hlstd{select.box} \hlkwb{<-} \hlkwd{switch}\hlstd{(input}\hlopt{$}\hlstd{metric,}
                         \hlstr{"Percent of California Consumption"} \hlstd{=} \hlnum{1L}\hlstd{,}
                         \hlstr{"Per Capita Consumption"} \hlstd{=} \hlnum{2L}\hlstd{)}

    \hlkwd{color.map}\hlstd{(}\hlkwc{color.variable} \hlstd{= select.box,} \hlkwc{year.map} \hlstd{=} \hlkwd{as.numeric}\hlstd{(input}\hlopt{$}\hlstd{year))}
  \hlstd{\})}

  \hlstd{output}\hlopt{$}\hlstd{countyText} \hlkwb{<-} \hlkwd{renderPrint}\hlstd{(\{}
    \hlkwd{cat}\hlstd{(}\hlstr{"That's "}\hlstd{,} \hlkwd{simpleCap}\hlstd{(}\hlkwd{theCounty}\hlstd{()),} \hlstr{" county."}\hlstd{)}

  \hlstd{\})}

  \hlstd{output}\hlopt{$}\hlstd{useagePlot} \hlkwb{=}
    \hlkwd{renderPlot}\hlstd{(}
      \hlkwd{gg.wrapper}\hlstd{(}\hlkwc{county.name} \hlstd{=} \hlkwd{theCounty}\hlstd{(),} \hlkwc{year.gg} \hlstd{=} \hlkwd{as.numeric}\hlstd{(input}\hlopt{$}\hlstd{year))}
    \hlstd{)}

  \hlstd{output}\hlopt{$}\hlstd{siteMap} \hlkwb{<-} \hlkwd{renderLeaflet}\hlstd{(\{}
    \hlkwd{leaflet}\hlstd{(}\hlkwc{data} \hlstd{= goodSurfaceData)} \hlopt
      \hlkwd{addProviderTiles}\hlstd{(}\hlstr{"Esri.WorldTopoMap"}\hlstd{)} \hlopt
      \hlkwd{addCircleMarkers}\hlstd{(}\hlopt{~}\hlstd{long,} \hlopt{~}\hlstd{lat,} \hlkwc{layerId} \hlstd{=} \hlopt{~} \hlstd{siteNumber,} \hlkwc{radius} \hlstd{=} \hlnum{2}\hlstd{)}
  \hlstd{\})}

  \hlkwd{observe}\hlstd{(\{}
    \hlkwa{if}\hlstd{(}\hlopt{!}\hlkwd{is.null}\hlstd{(input}\hlopt{$}\hlstd{siteMap_marker_click}\hlopt{$}\hlstd{id))}
      \hlstd{output}\hlopt{$}\hlstd{sitePlot} \hlkwb{=} \hlkwd{renderPlot}\hlstd{(\{}
        \hlkwd{plot.discharge}\hlstd{(}\hlkwc{siteNumber} \hlstd{= input}\hlopt{$}\hlstd{siteMap_marker_click}\hlopt{$}\hlstd{id)}
      \hlstd{\})}
  \hlstd{\})}

  \hlstd{output}\hlopt{$}\hlstd{gwMap} \hlkwb{<-} \hlkwd{renderLeaflet}\hlstd{(\{}
    \hlkwd{leaflet}\hlstd{(}\hlkwc{data} \hlstd{= gwSites)} \hlopt
      \hlkwd{addProviderTiles}\hlstd{(}\hlstr{"Esri.WorldTopoMap"}\hlstd{)} \hlopt
      \hlkwd{addCircleMarkers}\hlstd{(}\hlopt{~}\hlstd{long,} \hlopt{~}\hlstd{lat,} \hlkwc{layerId} \hlstd{=} \hlopt{~} \hlstd{siteNumber,}
                       \hlkwc{color} \hlstd{=} \hlstr{"red"}\hlstd{,} \hlkwc{radius} \hlstd{=} \hlnum{2}\hlstd{)}
  \hlstd{\})}

  \hlkwd{observe}\hlstd{(\{}
    \hlkwa{if}\hlstd{(}\hlopt{!}\hlkwd{is.null}\hlstd{(input}\hlopt{$}\hlstd{gwMap_marker_click}\hlopt{$}\hlstd{id))}
      \hlstd{output}\hlopt{$}\hlstd{GWPlot} \hlkwb{=} \hlkwd{renderPlot}\hlstd{(\{}
        \hlkwd{gwPlot}\hlstd{(input}\hlopt{$}\hlstd{gwMap_marker_click}\hlopt{$}\hlstd{id)}
      \hlstd{\})}
  \hlstd{\})}

  \hlkwa{if}\hlstd{(}\hlnum{FALSE}\hlstd{)\{}
    \hlstd{theGWSites} \hlkwb{=} \hlkwd{reactiveValues}\hlstd{()}
    \hlstd{theGWSites}\hlopt{$}\hlstd{sites} \hlkwb{=} \hlnum{374004122092106} \hlcom{# character(0)}

    \hlcom{# On gw-map click, if well isn't in the vector to be plotted, add it.}
    \hlkwd{observe}\hlstd{(\{}
      \hlkwa{if}\hlstd{(}\hlopt{!}\hlkwd{is.null}\hlstd{(input}\hlopt{$}\hlstd{gwMap_marker_click}\hlopt{$}\hlstd{id)} \hlopt{&}
           \hlopt{!}\hlstd{input}\hlopt{$}\hlstd{gwMap_marker_click}\hlopt{$}\hlstd{id} \hlopt \hlstd{theGWSites}\hlopt{$}\hlstd{sites) \{}
        \hlstd{nextWell} \hlkwb{<-} \hlkwd{isolate}\hlstd{(input}\hlopt{$}\hlstd{gwMap_marker_click}\hlopt{$}\hlstd{id)}
        \hlkwa{if}\hlstd{(}\hlopt{!}\hlstd{nextWell} \hlopt \hlstd{theGWSites}\hlopt{$}\hlstd{sites)}
          \hlkwd{isolate}\hlstd{(theGWSites}\hlopt{$}\hlstd{sites} \hlkwb{<-} \hlkwd{c}\hlstd{(theGWSites}\hlopt{$}\hlstd{sites, nextWell))}
      \hlstd{\}}
    \hlstd{\})}

    \hlkwd{observe}\hlstd{(\{}
      \hlkwa{if}\hlstd{(input}\hlopt{$}\hlstd{clear} \hlopt{>} \hlnum{0}\hlstd{) \{}
        \hlstd{theGWSites}\hlopt{$}\hlstd{sites} \hlkwb{=} \hlnum{374004122092106}  \hlcom{# ""}
      \hlstd{\}}
    \hlstd{\})}

    \hlstd{output}\hlopt{$}\hlstd{wellsInfo} \hlkwb{=}
      \hlkwd{renderPrint}\hlstd{(}\hlkwd{cat}\hlstd{(}\hlstr{"Plotting wells:"}\hlstd{, theGWSites}\hlopt{$}\hlstd{sites,} \hlkwc{sep} \hlstd{=} \hlstr{"\textbackslash{}n"}\hlstd{))}

    \hlstd{output}\hlopt{$}\hlstd{GWPlot} \hlkwb{=} \hlkwd{renderPlot}\hlstd{(\{}
      \hlkwd{gwPlot}\hlstd{(theGWSites}\hlopt{$}\hlstd{sites)}
    \hlstd{\})}

  \hlstd{\}}

\hlstd{\})}
\end{alltt}
\end{kframe}
\end{knitrout}

functions.R

\begin{knitrout}
\definecolor{shadecolor}{rgb}{0.969, 0.969, 0.969}\color{fgcolor}\begin{kframe}
\begin{alltt}
\hlcom{# install packages and load packages ####}
\hlstd{packages.list} \hlkwb{=} \hlkwd{c}\hlstd{(}\hlstr{"shiny"}\hlstd{,}
                  \hlstr{"maps"}\hlstd{,}
                  \hlstr{"mapdata"}\hlstd{,}
                  \hlstr{"maptools"}\hlstd{,}
                  \hlstr{"Hmisc"}\hlstd{,}
                  \hlstr{"ggplot2"}\hlstd{,}
                  \hlstr{"reshape2"}\hlstd{,}
                  \hlstr{"dataRetrieval"}\hlstd{,}
                  \hlstr{"data.table"}\hlstd{,}
                  \hlstr{"RColorBrewer"}\hlstd{)}

\hlkwa{for} \hlstd{(p} \hlkwa{in} \hlstd{packages.list) \{}
  \hlkwa{if} \hlstd{(}\hlopt{!}\hlstd{(p} \hlopt \hlkwd{rownames}\hlstd{(}\hlkwd{installed.packages}\hlstd{())))}
    \hlkwd{install.packages}\hlstd{(}\hlkwc{pkgs} \hlstd{= p)}
\hlstd{\}}

\hlkwa{if} \hlstd{(}\hlopt{!}\hlstd{(}\hlstr{"leaflet"} \hlopt \hlkwd{rownames}\hlstd{(}\hlkwd{installed.packages}\hlstd{()))) \{}
  \hlkwd{require}\hlstd{(}\hlstr{"devtools"}\hlstd{)}
  \hlstd{devtools}\hlopt{::}\hlkwd{install_github}\hlstd{(}\hlstr{"rstudio/leaflet"}\hlstd{)}
\hlstd{\}}

\hlkwd{library}\hlstd{(shiny)}
\hlkwd{library}\hlstd{(maps)}
\hlkwd{library}\hlstd{(leaflet)}
\hlkwd{library}\hlstd{(mapdata)}
\hlkwd{library}\hlstd{(maptools)}
\hlkwd{library}\hlstd{(Hmisc)}
\hlkwd{library}\hlstd{(ggplot2)}
\hlkwd{library}\hlstd{(reshape2)}
\hlkwd{library}\hlstd{(dataRetrieval)}
\hlkwd{library}\hlstd{(data.table)}

\hlstd{latlong2county} \hlkwb{<-} \hlkwa{function}\hlstd{(}\hlkwc{pointsDF}\hlstd{,} \hlkwc{wantState} \hlstd{=} \hlnum{FALSE}\hlstd{) \{}
  \hlcom{# Taken verbetim from http://stackoverflow.com/questions/13316185}
  \hlcom{# Prepare SpatialPolygons object with one SpatialPolygon}
  \hlcom{# per county}
  \hlstd{counties} \hlkwb{<-} \hlkwd{map}\hlstd{(}\hlstr{'county'}\hlstd{,} \hlkwc{fill}\hlstd{=}\hlnum{TRUE}\hlstd{,} \hlkwc{col}\hlstd{=}\hlstr{"transparent"}\hlstd{,} \hlkwc{plot}\hlstd{=}\hlnum{FALSE}\hlstd{)}
  \hlstd{IDs} \hlkwb{<-} \hlkwd{sapply}\hlstd{(}\hlkwd{strsplit}\hlstd{(counties}\hlopt{$}\hlstd{names,} \hlstr{":"}\hlstd{),} \hlkwa{function}\hlstd{(}\hlkwc{x}\hlstd{) x[}\hlnum{1}\hlstd{])}
  \hlstd{counties_sp} \hlkwb{<-} \hlkwd{map2SpatialPolygons}\hlstd{(counties,} \hlkwc{IDs}\hlstd{=IDs,}
                                     \hlkwc{proj4string}\hlstd{=}\hlkwd{CRS}\hlstd{(}\hlstr{"+proj=longlat +datum=wgs84"}\hlstd{))}

  \hlcom{# Convert pointsDF to a SpatialPoints object}
  \hlstd{pointsSP} \hlkwb{<-} \hlkwd{SpatialPoints}\hlstd{(pointsDF,}
                            \hlkwc{proj4string}\hlstd{=}\hlkwd{CRS}\hlstd{(}\hlstr{"+proj=longlat +datum=wgs84"}\hlstd{))}

  \hlcom{# Use 'over' to get _indices_ of the Polygons object containing each point}
  \hlstd{indices} \hlkwb{<-} \hlkwd{over}\hlstd{(pointsSP, counties_sp)}

  \hlcom{# Return the county names of the Polygons object containing each point}
  \hlstd{countyNames} \hlkwb{<-} \hlkwd{sapply}\hlstd{(counties_sp}\hlopt{@}\hlkwc{polygons}\hlstd{,} \hlkwa{function}\hlstd{(}\hlkwc{x}\hlstd{) x}\hlopt{@}\hlkwc{ID}\hlstd{)}

  \hlstd{county} \hlkwb{=} \hlkwd{strsplit}\hlstd{(countyNames[indices],} \hlstr{","}\hlstd{)[[}\hlnum{1}\hlstd{]]}

  \hlkwa{if}\hlstd{(wantState) \{ county}
  \hlstd{\}} \hlkwa{else} \hlstd{\{}
    \hlstd{county[}\hlnum{2}\hlstd{]}
  \hlstd{\}}
\hlstd{\}}

\hlstd{color.map} \hlkwb{=} \hlkwa{function}\hlstd{(}\hlkwc{color.variable} \hlstd{=} \hlnum{1L}\hlstd{,} \hlkwc{year.map}\hlstd{) \{}
  \hlcom{# adapted from the example in the help doc of "map()" function.}

  \hlcom{# manipulate the water dataset so that we can match it to the map data.}
  \hlstd{df.water.cal} \hlkwb{=} \hlkwd{water.consum.data}\hlstd{(}\hlkwc{long} \hlstd{=} \hlnum{FALSE}\hlstd{,} \hlkwc{year}\hlstd{= year.map)}
  \hlkwa{if}\hlstd{(year.map} \hlopt{==} \hlnum{2010}\hlstd{)\{}
  \hlstd{df.water.cal}\hlopt{$}\hlstd{cal.County} \hlkwb{=} \hlkwd{sapply}\hlstd{(}\hlkwc{X} \hlstd{=} \hlkwd{strsplit}\hlstd{(}\hlkwc{x} \hlstd{=} \hlkwd{levels}\hlstd{(df.water.cal}\hlopt{$}\hlstd{County),} \hlkwc{split} \hlstd{=} \hlstr{' County'}\hlstd{),}
                                   \hlkwc{FUN} \hlstd{=} \hlkwa{function}\hlstd{(}\hlkwc{x}\hlstd{) x[[}\hlnum{1}\hlstd{]])}
  \hlstd{\}} \hlkwa{else} \hlstd{\{}
    \hlstd{df.water.cal}\hlopt{$}\hlstd{cal.County}\hlkwb{=}\hlstd{df.water.cal}\hlopt{$}\hlstd{County}
  \hlstd{\}}
  \hlstd{df.water.cal}\hlopt{$}\hlstd{polyname} \hlkwb{=} \hlkwd{paste0}\hlstd{(}\hlstr{"california,"}\hlstd{,} \hlkwd{tolower}\hlstd{(df.water.cal}\hlopt{$}\hlstd{cal.County))}

  \hlcom{# define color buckets}
  \hlstd{colors} \hlkwb{=} \hlkwd{c}\hlstd{(}\hlstr{"#fee5d9"}\hlstd{,}
             \hlstr{"#fcbba1"}\hlstd{,}
             \hlstr{"#fc9272"}\hlstd{,}
             \hlstr{"#fb6a4a"}\hlstd{,}
             \hlstr{"#de2d26"}\hlstd{,}
             \hlstr{"#a50f15"}\hlstd{)}

  \hlkwa{if} \hlstd{(color.variable} \hlopt{==} \hlnum{1L}\hlstd{) \{}
    \hlstd{df.water.cal}\hlopt{$}\hlstd{colorBuckets} \hlkwb{<-} \hlkwd{as.numeric}\hlstd{(}\hlkwd{cut}\hlstd{(df.water.cal}\hlopt{$}\hlstd{Percent,} \hlkwc{breaks} \hlstd{=} \hlkwd{c}\hlstd{(}\hlnum{0}\hlstd{,} \hlnum{0.01}\hlstd{,} \hlnum{1}\hlopt{:}\hlnum{5}\hlopt{*}\hlnum{0.02}\hlstd{)))}
    \hlstd{leg.txt} \hlkwb{<-} \hlkwd{c}\hlstd{(}\hlstr{"<1%"}\hlstd{,} \hlstr{"1-2%"}\hlstd{,} \hlstr{"2-4%"}\hlstd{,} \hlstr{"4-6%"}\hlstd{,} \hlstr{"6-8%"}\hlstd{,} \hlstr{"8-10%"}\hlstd{)}
    \hlstd{title.txt} \hlkwb{<-} \hlstr{"County consumption:\textbackslash{}n% of California total"}
  \hlstd{\}} \hlkwa{else if} \hlstd{(color.variable} \hlopt{==} \hlnum{2L}\hlstd{) \{}
    \hlstd{df.water.cal}\hlopt{$}\hlstd{colorBuckets} \hlkwb{<-} \hlkwd{as.numeric}\hlstd{(}\hlkwd{cut}\hlstd{(df.water.cal}\hlopt{$}\hlstd{Per.Cap,} \hlkwc{breaks} \hlstd{=} \hlkwd{c}\hlstd{(}\hlnum{0}\hlstd{,} \hlnum{1}\hlstd{,} \hlnum{5}\hlstd{,} \hlnum{10}\hlstd{,} \hlnum{20}\hlstd{,} \hlnum{40}\hlstd{)))}
    \hlstd{leg.txt} \hlkwb{<-} \hlkwd{c}\hlstd{(}\hlstr{"<1"}\hlstd{,} \hlstr{"1-5"}\hlstd{,} \hlstr{"5-10"}\hlstd{,} \hlstr{"10-20"}\hlstd{,} \hlstr{">20"}\hlstd{)}
    \hlstd{title.txt} \hlkwb{<-} \hlstr{"Per Capita Consumption \textbackslash{}n (Mgal/day/1000 people)"}
  \hlstd{\}}


  \hlcom{# align data with map definitions by (partial) matching state,county}
  \hlcom{# names, which include multiple polygons for some counties}
  \hlstd{colorsmatched} \hlkwb{<-} \hlstd{df.water.cal}\hlopt{$}\hlstd{colorBuckets[}
        \hlkwd{match}\hlstd{(}\hlkwd{map}\hlstd{(}\hlkwc{database} \hlstd{=} \hlstr{"county"}\hlstd{,} \hlkwc{regions} \hlstd{=} \hlstr{"california"}\hlstd{,} \hlkwc{plot}\hlstd{=}\hlnum{FALSE}\hlstd{)}\hlopt{$}\hlstd{names,df.water.cal}\hlopt{$}\hlstd{polyname)]}

  \hlcom{# draw map}
  \hlkwd{map}\hlstd{(}\hlkwc{database} \hlstd{=} \hlstr{"county"}\hlstd{,} \hlkwc{regions} \hlstd{=} \hlstr{"california"}\hlstd{,} \hlkwc{col} \hlstd{= colors[colorsmatched],} \hlkwc{fill} \hlstd{=} \hlnum{TRUE}\hlstd{,}
        \hlkwc{resolution} \hlstd{=} \hlnum{0}\hlstd{,} \hlkwc{lty} \hlstd{=} \hlnum{1}\hlstd{)}
  \hlcom{# the following lines might be useful if we draw the map for the whole U.S.}
  \hlcom{#   map("state", col = "white", fill = FALSE, add = TRUE, lty = 1, lwd = 0.2,}
  \hlcom{#       projection="polyconic")}

  \hlcom{#INCLUDING CITIES}
  \hlcom{#data(us.cities)}
  \hlcom{#cities <- c("^San Francisco", "^West Sacramento", "^Los Angeles", "^San Diego", "^Fresno")}
  \hlcom{#city.index <- sapply(cities , function(x)\{grep(x, us.cities[ ,1])\})}
  \hlcom{#map.cities(us.cities[city.index, ], country = "CA", }
        \hlcom{#label = TRUE, pch = 16, col = "black", cex = 1.5, font = 2)}
  \hlkwd{title}\hlstd{(title.txt)}
  \hlkwd{legend}\hlstd{(}\hlstr{"topright"}\hlstd{, leg.txt,} \hlkwc{fill} \hlstd{= colors)}
\hlstd{\}}

\hlstd{simpleCap} \hlkwb{<-} \hlkwa{function}\hlstd{(}\hlkwc{x}\hlstd{) \{}
  \hlcom{# From toupper() help file.}
  \hlstd{s} \hlkwb{<-} \hlkwd{strsplit}\hlstd{(x,} \hlstr{" "}\hlstd{)[[}\hlnum{1}\hlstd{]]}
  \hlkwd{paste}\hlstd{(}\hlkwd{toupper}\hlstd{(}\hlkwd{substring}\hlstd{(s,} \hlnum{1}\hlstd{,}\hlnum{1}\hlstd{)),} \hlkwd{substring}\hlstd{(s,} \hlnum{2}\hlstd{),}
        \hlkwc{sep}\hlstd{=}\hlstr{""}\hlstd{,} \hlkwc{collapse}\hlstd{=}\hlstr{" "}\hlstd{)}
\hlstd{\}}
\end{alltt}
\end{kframe}
\end{knitrout}

plot.R

\begin{knitrout}
\definecolor{shadecolor}{rgb}{0.969, 0.969, 0.969}\color{fgcolor}\begin{kframe}
\begin{alltt}
\hlcom{#This Script will make the ggplot wrapper}
\hlcom{# rm(list=ls())  # ML: It seems very dangerous to me to source files with this command.}

\hlcom{# We want to build this data.frame once, not on each call to gg.wrapper.}
\hlstd{water.consum.data} \hlkwb{<-} \hlkwa{function} \hlstd{(}\hlkwc{select.variables} \hlstd{=} \hlkwa{NULL}\hlstd{,}
                               \hlkwc{long}\hlstd{,}

                               \hlkwc{year}\hlstd{=}\hlnum{2010}\hlstd{) \{}
  \hlstd{df} \hlkwb{<-} \hlkwd{read.csv}\hlstd{(}\hlkwc{file} \hlstd{=} \hlkwd{paste0}\hlstd{(}\hlstr{"./clean_data/ca_"}\hlstd{,year,}\hlstr{".csv"}\hlstd{))}


  \hlkwa{if} \hlstd{(}\hlopt{!}\hlkwd{is.null}\hlstd{(select.variables))}
    \hlstd{df} \hlkwb{<-} \hlstd{df[ , select.variables]}

  \hlcom{#To use ggplot, I need to reshape the data from wide to long.}
  \hlkwa{if} \hlstd{(long) \{}
    \hlstd{df.long} \hlkwb{<-} \hlkwd{melt}\hlstd{(df,} \hlkwc{id} \hlstd{=} \hlstr{"County"}\hlstd{)}
    \hlkwd{names}\hlstd{(df.long)[}\hlnum{2}\hlopt{:}\hlnum{3}\hlstd{]} \hlkwb{<-} \hlkwd{c}\hlstd{(}\hlstr{"Source"}\hlstd{,}\hlstr{"Water"}\hlstd{)}
    \hlkwd{return}\hlstd{(df.long)}
  \hlstd{\}} \hlkwa{else} \hlstd{\{}
    \hlkwd{return} \hlstd{(df)}
  \hlstd{\}}
\hlstd{\}}

\hlstd{cc} \hlkwb{=} \hlkwd{rep}\hlstd{(}\hlstr{"NULL"}\hlstd{,} \hlnum{20}\hlstd{)}
\hlstd{cc[}\hlnum{3}\hlstd{]} \hlkwb{=} \hlstr{"character"}
\hlstd{counties} \hlkwb{=} \hlkwd{tolower}\hlstd{(}\hlkwd{read.csv}\hlstd{(}\hlstr{"./clean_data//ca_2010.csv"}\hlstd{,} \hlkwc{colClasses} \hlstd{= cc)}\hlopt{$}\hlstd{County)}

\hlcom{# debugonce(gg.wrapper)}
\hlcom{# gg.wrapper("imperial", "2000")}
\hlcom{#plot}
\hlstd{gg.wrapper} \hlkwb{<-} \hlkwa{function}\hlstd{(}\hlkwc{county.name}\hlstd{,} \hlkwc{year.gg}\hlstd{)\{}

  \hlstd{theDF} \hlkwb{<-} \hlkwd{water.consum.data}\hlstd{(}
    \hlkwc{select.variables} \hlstd{=} \hlkwd{c}\hlstd{(}\hlstr{"County"}\hlstd{,} \hlstr{"Public.Supply"}\hlstd{,} \hlstr{"Domestic.Self"}\hlstd{,} \hlstr{"Industry"}\hlstd{,}
                         \hlstr{"Irrigation"}\hlstd{,} \hlstr{"Livestock"}\hlstd{,}
                         \hlstr{"Aquaculture"}\hlstd{,} \hlstr{"Mining"}\hlstd{,}\hlstr{"Thermoelectric"}\hlstd{),}
    \hlkwc{long} \hlstd{=} \hlnum{TRUE}\hlstd{,}
    \hlkwc{year} \hlstd{= year.gg)}

  \hlstd{plotDF} \hlkwb{=} \hlstd{theDF[}\hlkwd{grep}\hlstd{(}\hlkwc{pattern} \hlstd{= county.name,} \hlkwc{x} \hlstd{=} \hlkwd{tolower}\hlstd{(theDF}\hlopt{$}\hlstd{County)),]}
  \hlstd{plotDF}\hlopt{$}\hlstd{Water} \hlkwb{=} \hlstd{plotDF}\hlopt{$}\hlstd{Water} \hlopt{*} \hlnum{.325851} \hlopt{*} \hlnum{365.25} \hlcom{# MGal/day -> AF/year}
\hlcom{#  plotDF = plotDF[plotDF$Water > 0, ]}
  \hlstd{plotDF}\hlopt{$}\hlstd{lab} \hlkwb{=} \hlkwd{sprintf}\hlstd{(}\hlstr{"%1.0f"}\hlstd{, plotDF}\hlopt{$}\hlstd{Water)}
  \hlstd{plotDF}\hlopt{$}\hlstd{Source} \hlkwb{=} \hlkwd{factor}\hlstd{(plotDF}\hlopt{$}\hlstd{Source,}
                         \hlkwc{levels} \hlstd{= plotDF}\hlopt{$}\hlstd{Source[}\hlkwd{order}\hlstd{(plotDF}\hlopt{$}\hlstd{Water)],}
                         \hlkwc{ordered} \hlstd{=} \hlnum{TRUE}\hlstd{)}
  \hlcom{#First I will subsample the data.  Some data is a double count.  }
  \hlcom{#For example Ir=Ir.C+Ir.G (i.e. Irrigation = Irrigation Crops + Irrigation Golf)}
  \hlstd{col} \hlkwb{=} \hlkwd{brewer.pal}\hlstd{(}\hlnum{8}\hlstd{,} \hlstr{"Set1"}\hlstd{)}
  \hlkwd{names}\hlstd{(col)} \hlkwb{=} \hlkwd{c}\hlstd{(}\hlstr{"Public.Supply"}\hlstd{,} \hlstr{"Domestic.Self"}\hlstd{,} \hlstr{"Industry"}\hlstd{,}
                 \hlstr{"Irrigation"}\hlstd{,} \hlstr{"Livestock"}\hlstd{,}
                 \hlstr{"Aquaculture"}\hlstd{,} \hlstr{"Mining"}\hlstd{,}\hlstr{"Thermoelectric"}\hlstd{)}

  \hlstd{plot.water} \hlkwb{<-} \hlkwd{ggplot}\hlstd{(}\hlkwc{data} \hlstd{= plotDF,} \hlkwd{aes}\hlstd{(}\hlkwc{x}\hlstd{=Source,}\hlkwc{y}\hlstd{=Water,} \hlkwc{fill} \hlstd{=Source ))}\hlopt{+}
    \hlkwd{geom_bar}\hlstd{(}\hlkwc{stat}\hlstd{=}\hlstr{"identity"}\hlstd{)}\hlopt{+}
    \hlkwd{theme_bw}\hlstd{()}\hlopt{+}
\hlcom{#    scale_y_continuous("Total Fresh Water Withdrawn (Mgal/day)")+}
    \hlkwd{scale_x_discrete}\hlstd{(}\hlstr{""}\hlstd{)} \hlopt{+}
    \hlkwd{scale_y_log10}\hlstd{(}\hlstr{"Fresh Water Use (acre-feet/year)"}\hlstd{)} \hlopt{+}
\hlcom{#                  limits = c(1, max(theDF$Water) + .05 * max(theDF$Water))) +}
  \hlkwd{scale_fill_manual}\hlstd{(}\hlkwc{values} \hlstd{= col)} \hlopt{+}
  \hlkwd{geom_text}\hlstd{(}\hlkwd{aes}\hlstd{(}\hlkwc{label} \hlstd{= lab,} \hlkwc{y} \hlstd{=} \hlkwd{ifelse}\hlstd{(Water} \hlopt{<} \hlnum{2}\hlstd{,} \hlnum{1}\hlstd{, Water} \hlopt{/} \hlnum{2}\hlstd{)))} \hlopt{+}
    \hlkwd{coord_flip}\hlstd{()}\hlopt{+}
    \hlkwd{guides}\hlstd{(}\hlkwc{fill}\hlstd{=}\hlnum{FALSE}\hlstd{)}\hlopt{+}
    \hlkwd{ggtitle}\hlstd{(}\hlkwd{paste}\hlstd{(}\hlkwd{simpleCap}\hlstd{(county.name),} \hlstr{"County"}\hlstd{))}\hlopt{+}
    \hlkwd{theme}\hlstd{(}\hlkwc{plot.title} \hlstd{=} \hlkwd{element_text}\hlstd{(}\hlkwc{size}\hlstd{=}\hlnum{18}\hlstd{,} \hlkwc{face}\hlstd{=}\hlstr{"bold"}\hlstd{),}
                \hlcom{#Don't adjust text size. If you increase it will cut off San Luis Obispo County}
          \hlkwc{axis.text.y} \hlstd{=} \hlkwd{element_text}\hlstd{(}\hlkwc{size} \hlstd{=} \hlnum{15}\hlstd{),}
          \hlkwc{axis.text.x} \hlstd{=} \hlkwd{element_text}\hlstd{(}\hlkwc{size} \hlstd{=} \hlnum{12}\hlstd{),}
          \hlkwc{axis.title.x} \hlstd{=} \hlkwd{element_text}\hlstd{(}\hlkwc{size} \hlstd{=} \hlnum{15}\hlstd{))}
  \hlkwd{return}\hlstd{(plot.water)}
\hlstd{\}}
\end{alltt}
\end{kframe}
\end{knitrout}

USGSplot.R

\begin{knitrout}
\definecolor{shadecolor}{rgb}{0.969, 0.969, 0.969}\color{fgcolor}\begin{kframe}
\begin{alltt}
  plot = \hlkwd{try}(\hlkwd{plot.discharge}(siteNumber = siteNumber))

  \hlkwd{require}(ggplot2)
  null.plot = \hlkwd{qplot}(x = 0, y = 0)

  \hlkwd{if} ((\hlkwd{class}(plot)[[1]] != \hlstr{"try-error"})) \{
    \hlkwd{return}(\hlkwd{list}(0, plot))
  \} else \{
    \hlkwd{return}(\hlkwd{list}(1, null.plot))
  \}
\}

plot.discharge.2 = \hlkwd{function}(siteNumber,
                            startDate = NULL,
                            endDate = NULL) \{
  \hlkwd{require}(dataRetrieval)

  surfaceData = \hlkwd{try}(\hlkwd{readNWISmeas}(siteNumbers = siteNumber))

  \hlkwd{stopifnot}(\hlstr{"measurement_dt"} %in% \hlkwd{names}(surfaceData))
  \hlkwd{stopifnot}(\hlstr{"discharge_va"} %in% \hlkwd{names}(surfaceData))
  \hlkwd{stopifnot}(\hlstr{"gage_height_va"} %in% \hlkwd{names}(surfaceData))

  siteInfo = \hlkwd{attr}(surfaceData, \hlstr{"siteInfo"})

  \hlkwd{if} (!\hlkwd{is.null}(startDate))
    surfaceData = surfaceData[surfaceData$measurement_dt >= \hlkwd{as.POSIXct}(startDate), ]

  \hlkwd{if} (!\hlkwd{is.null}(endDate))
    surfaceData = surfaceData[surfaceData$measurement_dt <= \hlkwd{as.POSIXct}(endDate), ]

  \hlkwd{require}(ggplot2)
  plot = \hlkwd{ggplot}(data = surfaceData, \hlkwd{aes}(x = measurement_dt, y = discharge_va)) +
    \hlkwd{geom_point}() +
    \hlkwd{geom_line}() +
    \hlkwd{xlab}(\hlstr{"Date"}) +
    \hlkwd{ylab}(\hlstr{"Discharge, cubic feet per second"}) +
    \hlkwd{ggtitle}(siteInfo$station_nm)

  \hlkwd{return}(plot)
\}


plot.discharge = \hlkwd{function}(siteNumber,
                          parameterCd = \hlstr{"00060"}) \{
  \hlkwd{require}(dataRetrieval)

  data = \hlkwd{readNWISdv}(siteNumber, startDate = \hlstr{"2011-10-01"},
                    parameterCd)

  data = \hlkwd{renameNWISColumns}(data)

  \hlkwd{if}(!\hlstr{"Flow"} %in% \hlkwd{names}(data))
    \hlkwd{stop}(\hlstr{"Sorry, USGS doesn't have flow data for that station."})
     

  variableInfo = \hlkwd{attr}(data, \hlstr{"variableInfo"})
  siteInfo = \hlkwd{attr}(data, \hlstr{"siteInfo"})

  \hlkwd{require}(ggplot2)
 \hlkwd{ggplot}(data = data, \hlkwd{aes}(x = Date, y = Flow)) +
    \hlkwd{geom_line}() +
    \hlkwd{xlab}(\hlstr{"Date"}) +
    \hlkwd{ylab}(variableInfo$parameter_desc) +
    \hlkwd{theme_bw}() +
    \hlkwd{scale_y_log10}() +
    \hlkwd{ggtitle}(\hlkwd{paste0}(\hlstr{"Daily warter discharge at: "}, siteInfo$station_nm))

\hlcom{#  return(plot)}
\}

gwPlot = \hlkwd{function}(siteNum)
\{
  \hlkwd{ggplot}(gwLevels[gwLevels$siteNumber == siteNum, ], 
         \hlkwd{aes}(x = date, 
\hlcom{#             y = log10(level))) +}
             y = level)) +
\hlcom{             # color = siteNumber)) +}
    \hlkwd{scale_y_reverse}() + 
    \hlkwd{scale_x_date}(limits = \hlkwd{c}(\hlkwd{as.Date}(\hlstr{"2011-10-01"}), \hlkwd{Sys.Date}())) +
    \hlkwd{geom_line}() +
    \hlkwd{geom_point}() +
\hlcom{#    ylab("Depth, log10(feet)") +}
    \hlkwd{ylab}(\hlstr{"\hlkwd{Depth} (feet)"}) +
    \hlkwd{xlab}(\hlstr{"Date"}) +
    \hlkwd{theme_bw}()
\}
\end{alltt}
\end{kframe}
\end{knitrout}

readUSGSData.R

\begin{knitrout}
\definecolor{shadecolor}{rgb}{0.969, 0.969, 0.969}\color{fgcolor}\begin{kframe}
\begin{alltt}
\hlstd{sites} \hlkwb{=} \hlkwd{readRDS}\hlstd{(}\hlstr{"clean_data/allSites.RDS"}\hlstd{)}
\hlstd{goodSurfaceData} \hlkwb{=} \hlkwd{readRDS}\hlstd{(}\hlstr{"clean_data/goodSurfaceSites.RDS"}\hlstd{)}
\hlstd{gwLevels} \hlkwb{=} \hlkwd{readRDS}\hlstd{(}\hlstr{"clean_data/gwLevels.RDS"}\hlstd{)}
\hlstd{gwSites} \hlkwb{=} \hlstd{sites[sites}\hlopt{$}\hlstd{siteNumber} \hlopt \hlstd{gwLevels}\hlopt{$}\hlstd{siteNumber, ]}
\end{alltt}
\end{kframe}
\end{knitrout}
\end{document}
